

% معرفی تمپلیت
\section*{راهنمای استفاده از تمپلیت}

این تمپلیت برای نوشتن تمرین‌ها و گزارش‌های ریاضی به‌ویژه در دانشکده ریاضی و علوم کامپیوتر دانشگاه صنعتی امیرکبیر طراحی شده است. در اینجا نحوه استفاده از تمپلیت شرح داده شده است.

\subsection*{1. تنظیمات اولیه}

در ابتدا باید فایل \texttt{configs.tex} را برای شخصی‌سازی اطلاعات خود و دوره‌ای که در آن هستید ویرایش کنید. در این فایل می‌توانید موارد زیر را تنظیم کنید:
\begin{itemize}
    \item \textbf{نام و نام‌خانوادگی} خود را در دستور \texttt{\textbackslash myname} وارد کنید.
    \item \textbf{شماره دانشجویی} خود را در دستور \texttt{\textbackslash mystdID} وارد کنید.
    \item \textbf{نام استاد} را در دستور \texttt{\textbackslash type} وارد کنید.
    \item \textbf{نام دوره} و \textbf{ترم} را نیز می‌توانید در دستور \texttt{\textbackslash semester} تنظیم کنید.
\end{itemize}

\subsection*{2. نوشتن تمرین‌ها و راه‌حل‌ها}

برای نوشتن یک تمرین جدید، از محیط \texttt{prob} استفاده کنید:
\begin{LTR}
    \begin{verbatim}
    \begin{prob}
        Write your problem here.
    \end{prob}
    \end{verbatim}
\end{LTR}

برای نوشتن راه‌حل، از محیط \texttt{sol} استفاده کنید:

\begin{LTR}
    \begin{verbatim}
    \begin{sol}
        write your solution here.
    \end{sol}
    \end{verbatim}
\end{LTR}

در نهایت، برای فرمت‌دهی صحیح، تمپلیت به‌صورت خودکار از دستورات و تنظیمات سفارشی‌شده استفاده می‌کند.

\subsection*{3. استفاده از محیط‌های دیگر}

اگر نیاز به نوشتن اثبات ریاضی یا توضیحات اضافی دارید، می‌توانید از محیط‌های زیر استفاده کنید:

\begin{itemize}
    \item \texttt{proof} برای نوشتن اثبات‌های ریاضی.
    \item \texttt{customEnv} برای محیط‌های سفارشی که می‌خواهید در کنار شماره‌گذاری‌های خاص داشته باشید.
\end{itemize}

برای مثال:

\begin{LTR}
    \begin{verbatim}
    \begin{proof}
        Write your proof here.
    \end{proof}
    \end{verbatim}
\end{LTR}

\subsection*{4. افزودن کدهای برنامه‌نویسی}

اگر نیاز دارید که کدهای برنامه‌نویسی را در متن خود نمایش دهید، می‌توانید از محیط \texttt{lstlisting} استفاده کنید. به عنوان مثال:

\begin{LTR}
    \begin{verbatim}
    \begin{lstlisting}[language=Python]
    def solve_integral():
        return 1 / 3
    \end{lstlisting}
    \end{verbatim}
\end{LTR}


این کد به‌صورت خودکار با رنگ‌های مختلف و فرمت مناسب در LaTeX نمایش داده خواهد شد.

% اضافه کردن مثال حل‌شده
\section*{مثال حل‌شده}

در این قسمت، یک مثال کامل از نوشتن یک تمرین و راه‌حل آن آورده شده است:

\begin{prob}
    محاسبه انتگرال زیر:
    \[
        \int_{0}^{1} x^2 \, dx
    \]
\end{prob}

\begin{sol}
    برای محاسبه این انتگرال، از قاعده‌ی انتگرال‌گیری استفاده می‌کنیم:
    \[
        \int_{0}^{1} x^2 \, dx = \left[ \frac{x^3}{3} \right]_{0}^{1} = \frac{1}{3} - 0 = \frac{1}{3}
    \]
\end{sol}

\begin{proof}
    اثبات محاسبه این انتگرال به سادگی انجام می‌شود، زیرا فرمول مورد استفاده برای انتگرال‌گیری به‌طور مستقیم به‌دست می‌آید.
\end{proof}

\newpage

% اضافه کردن منابع
\section*{منابع}
در این قسمت می‌توانید منابع خود را ذکر کنید. به عنوان مثال:

\begin{itemize}
    \item کتاب مرجع: \textit{ریاضیات کاربردی}، نویسنده: دکتر علی‌اکبر نظری
    \item مقالات علمی مرتبط
\end{itemize}